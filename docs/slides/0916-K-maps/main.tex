\documentclass{article}
\usepackage[utf8]{inputenc}
\usepackage[top=1in]{geometry}
\usepackage{graphicx}
\usepackage{booktabs}
\usepackage{amsmath}
\usepackage{amsthm}
\usepackage[only]{excludeonly}
\usepackage{tikz}
\usetikzlibrary{circuits.logic.US,positioning,calc} 
\input{karnaugh}
\input{sym}
\title{Logic minimization: Minimum-cost circuits}
\author{Vikas Dhiman for ECE275}
\newtheorem{example}{Example}
\newtheorem{prob}{Problem}
\newtheorem{remark}{Remark}

\newcommand{\bx}{\bar{x}}
\newcommand{\by}{\bar{y}}
\newcommand{\bz}{\bar{z}}
\newcommand{\bA}{\bar{A}}
\newcommand{\bB}{\bar{B}}
\newcommand{\bC}{\bar{C}}

\begin{document}
\maketitle

\section{A few more Boolean problems}
\begin{example}
  Simplify the following Boolean expression:
  \[ f = x_1\bx_3 \bx_4 + x_2 \bx_3 \bx_4 + x_1 \bx_2 \bx_3 \]
\end{example}

\begin{example}
  Using algebraic manipulation to prove that:
  \[ x + yz = (x + y)(y+z) \]
\end{example}

\begin{example}
  Design a three-way light control
  \includegraphics[width=\linewidth]{figures/design-a-3-way-light-switch.pdf}
\end{example}

\section{Logic minimization}

A general optimization criteria for multi-level logic are to Minimize
some combination of:
\begin{enumerate}
\item Area occupied by the logic gates and interconnect;
\item the Critical Path Delay of the longest path through the logic;
\item the Degree of Testability of the circuit, measured in terms of the percentage
of faults covered by a specified set of test vectors, for an appropriate fault model
(Eg., single stuck faults, multiple stuck faults, etc.);
\item Power consumed by the logic gates.
\end{enumerate}

In this course, we will start with two-level multi-input circuits and a criteria
based on the number of gates/transistors/diodes.

\section{Programmable Logic Arrays}
\includegraphics[width=0.5\linewidth]{figures/PLA-abstract.png}
\includegraphics[width=0.5\linewidth]{figures/PLA-logic.png}

\section{Two-level circuits}
The cost that we are going to consider in this class depend upon:
\begin{enumerate}
\item Number of gates.
\item Number of input to the gates.
\end{enumerate}
More gates need more transistors, more area on the chip. More-inputs the gate
need more transistors within each gate. Number of gate inputs can be considered
secondary criterion to the number of gates.

\begin{example}
  Find the cost of the following Boolean expression $X = \bA\bB C + AB\bC + A\bB$.
\end{example}



%\bibliography{main}
%\bibliographystyle{plain}
\end{document}
