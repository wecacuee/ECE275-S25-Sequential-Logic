\documentclass[options]{article}
\usepackage{enumitem,amssymb}
\newlist{todolist}{itemize}{2}
\setlist[todolist]{label=$\square$}

\usepackage{pifont}
\newcommand{\cmark}{\ding{51}}%
\newcommand{\xmark}{\ding{55}}%
\newcommand{\done}{\rlap{$\square$}{\raisebox{2pt}{\large\hspace{1pt}\cmark}}%
  \hspace{-2.5pt}}
\newcommand{\wontfix}{\rlap{$\square$}{\large\hspace{1pt}\xmark}}


\begin{document}

\section{Study guide}
\subsection{Midterm 1}
\begin{todolist}
  \item[\done] Binary numbers, Hexadecimal, Sign-magnitude, One's-complement and
    Two's complement. Conversions between them.
    \begin{enumerate}
      \item Homework 1 and Lectures 08/31 and 09/02.
    \end{enumerate}
  \item[\done] Generate minterms, maxterms, SOP canonical form and POS
    canonical forms and convert between them\\
  \begin{enumerate}
    \item Lecture 09/09
  \end{enumerate}
  \item[\done]  Understand and use the laws and theorems of Boolean Algebra
  \begin{enumerate}
    \item Homework 2 and Lectures 09/16-09/19
  \end{enumerate}
  \item[\done]  Perform algebraic simplification using Boolean algebra
  \begin{enumerate}
    \item Homework 2 and Lectures 09/16-09/19
  \end{enumerate}
  \item[\done]  Simplification using K-maps
    \begin{enumerate}
    \item Homework 2 and 3 and Lectures 09/12-09/14
    \end{enumerate}
  \item[\done]  Derive sum of product and product of sums expressions for a combinational circuit
    \begin{enumerate}
    \item Homework 2 and 3 and Lectures 09/12-09/23
    \end{enumerate}
  \item[\done]  Convert combinational logic to NAND-NAND and NOR-NOR forms
    \begin{enumerate}
    \item Homework 3 and Lecture 09/28
    \end{enumerate}
\end{todolist}

\subsection{Midterm 2}
\begin{todolist}
  \item[\done]  Simplification using Quine-McCluskey method
    \begin{enumerate}
    \item Lecture 09/28 
    \end{enumerate}
  \item[\done]  Design combinational circuits for positive and negative logic
  \begin{enumerate}
    \item For Negative logic is H = 0, L = 0. See Example 6, on lecture 10/19
  \end{enumerate}
  \item[\done]  Design Hazard-free two level circuits.
  \begin{enumerate}
    \item See Example 14, on lecture 10/24
  \end{enumerate}
  \item[\done] Compute noise margin of one device
  \begin{enumerate}
    \item See Section 2 of lecture 10/17
   \end{enumerate}
  \item[\done] Describe how tri-state and open-collector outputs are different from totem-pole outputs.
    \begin{enumerate}
      \item See Definitions 11-13 covered in lecture 10/21
    \end{enumerate}
  \item[\done] Different between and limitations of level-triggered latches and edge-triggered flip-flops.
  \begin{enumerate}
    \item See lecture 10/26-10/28
  \end{enumerate}
\item[\done] Understand the difference between synchronous and asynchronous inputs
  \begin{enumerate}
  \item See lecture 10/26-10/28
  \end{enumerate}

\item Derive a state graph or state table from a word description of the problem
  \begin{enumerate}
  \item Lecture 11/02
  \end{enumerate}
\item Implement a design using JK, SR, D or T flip-flops
  \begin{enumerate}
  \item Lecture 10/02
  \end{enumerate}
\item Analyze a sequential circuit and derive a state-table and a state-graph
  \begin{enumerate}
  \item Lecture 11/04
  \end{enumerate}
\end{todolist}
\subsection{Final (includes previous topics)}
\begin{todolist}
  \item Compute fan out and noise margin of one device driving the same time
  \item Know the differences and similarities between PAL, PLA, and ROMs and can use each for logic design
  \item Design combinational circuits using multiplexers and decoders
  \item Reduce the number of states in a state table using row reduction and implication tables
  \item Perform a state assignment using the guideline method
  \item Analyse and design both Mealy and Moore sequential circuits with multiple inputs and multiple outputs
  \item Convert between Mealy and Moore designs
  \item Partition a system into multiple state machines
\end{todolist}

% \subsection{Labs}
% \begin{todolist}
%   \item[\done] Use computer tools to enter designs graphically and HDL
%   \item Simulate designs using computer tools
%   \item Use computer tools to program gate arrays logic and debug and test
% \end{todolist}

\end{document}