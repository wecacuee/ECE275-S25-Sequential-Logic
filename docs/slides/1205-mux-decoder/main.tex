\documentclass{article}
\usepackage[utf8]{inputenc}
\usepackage[top=1in]{geometry}
\usepackage{graphicx}
\usepackage{booktabs}
\usepackage{amsmath}
\usepackage{amsthm}
\usepackage{hyperref}
\usepackage{wrapfig}
\usepackage[only]{excludeonly}
\usepackage{tikz}
\usetikzlibrary{circuits.logic.US,positioning,calc} 
\usepackage[american]{circuitikz}

\input{karnaugh}
\input{sym}
\author{Vikas Dhiman for ECE275}
\newtheorem{example}{Example}
\newtheorem{prob}{Practice Problem}
\newtheorem{remark}{Remark}
\newtheorem{definition}{Definition}

\newcommand{\bx}{\bar{x}}
\newcommand{\by}{\bar{y}}
\newcommand{\bz}{\bar{z}}
\newcommand{\bA}{\bar{A}}
\newcommand{\bB}{\bar{B}}
\newcommand{\bC}{\bar{C}}
\newcommand{\bD}{\bar{D}}

\newcommand{\notescol}{white}
\title{Sequential logic design}
\begin{document}

\maketitle


\section{Objectives}
\begin{enumerate}
   \item Design combinational circuits using multiplexers and decoders
\end{enumerate}

\section{Design combinational circuit using multiplexers ~\cite[Section~2.8.1]{harris2022digital}}

\subsection{Review: 2to1 Multiplexer (MUX)}
\includegraphics[width=0.3\linewidth]{./media/harrisfig2.54-2to1-mux-symb-tt.png}
\includegraphics[width=0.3\linewidth]{./media/harrisfig2.55-2to1-mux.png}
\includegraphics[width=0.3\linewidth]{./media/harrisfig2.56-2to1-mux-tristate-buffs.png}

\subsection{Wider multiplexers}
Draw the symbol for a 4:1 MUX, an 8:1 MUX and a $2^N:1$ MUX and write
corresponding Boolean expressions.
\vspace{10em}


\begin{example}
  Design a circuit for $Y = A\bB + \bB \bC + \bA B C$ using a 8:1 MUX.
\end{example}
\vspace{10em}

\begin{remark}
  A $2^N:1$ MUX can be used to program any N-input logic function.
\end{remark}

\begin{example}
  Design a circuit for $Y = A\bB + \bB \bC + \bA B C$ using a 4:1 MUX and 
  NOT gates only.
\end{example}
\vspace{10em}

\begin{remark}
  A $2^{N-1}:1$ MUX can be used to program any N-input logic function, if we use
  literals on the input side.
\end{remark}

\begin{example}
  Design a circuit for $Y = \bA C + \bA B + B \bD $ using a 8:1 MUX and NOT
  gates only. Also design using 4:1 MUX and other gates.
   fewest gates.
\end{example}
\vspace{10em}

\section{Encoders and Decoders}

\begin{example}
Draw the symbol and the truth table for 2:4 decoder. Also write the logic expressions.
\end{example}
\vspace{10em}

\begin{example}
  Draw the symbol and the truth table for 3:8 decoder, 4:16 decoder and $N:2^N$ decoder. Also write the logic expressions.
\end{example}
\vspace{10em}

\begin{example}
Design a circuit for a XOR gate using a 2:4 decoder and an OR gate.
\end{example}
\vspace{10em}

\begin{example}
  Design a circuit for $Y = A\bB + \bB \bC + \bA B C$ using a 3:8 decoder and an
  OR gate.
\end{example}
\vspace{10em}


\subsection{Encoders}

\begin{example}
  Draw symbol and truth table for a 4:2 priority encoder. 
\end{example}
\vspace{10em}

\begin{example}
  Draw symbol and truth table for a 8:3 priority encoder. 
\end{example}
\vspace{10em}

\bibliography{main}
\bibliographystyle{plain}
\end{document}
