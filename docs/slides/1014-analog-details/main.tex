\documentclass{article}

\usepackage[utf8]{inputenc}
\usepackage[top=1in]{geometry}
\usepackage{graphicx}
\usepackage{booktabs}
\usepackage{amsmath}
\usepackage{amsthm}
\usepackage[only]{excludeonly}
\usepackage{tikz}
\usetikzlibrary{circuits.logic.US,positioning,calc} 
\usepackage[american]{circuitikz}

\input{karnaugh}
\input{sym}
\author{Vikas Dhiman for ECE275}
\newtheorem{example}{Example}
\newtheorem{prob}{Problem}
\newtheorem{remark}{Remark}
\newtheorem{definition}{Definition}

\newcommand{\bx}{\bar{x}}
\newcommand{\by}{\bar{y}}
\newcommand{\bz}{\bar{z}}
\newcommand{\bA}{\bar{A}}
\newcommand{\bB}{\bar{B}}
\newcommand{\bC}{\bar{C}}
\newcommand{\bD}{\bar{D}}

\newcommand{\notescol}{white}
\title{Analog details behind the digital abstraction}
\begin{document}

\maketitle

Some of the material is out of the textbook. Additional resources include
Appendix B of Brown and Vranesic book, ``Fundamentals of digital logic.''

\section{Objectives}
\begin{enumerate}
\item Describe how tri-state and open-collector outputs are different from totem-pole outputs
\item Compute noise margin of one device driving the same time
\end{enumerate}

\section{FPGA~\cite[Section~B.6.5]{stephen2022fundamentals}}
\includegraphics[width=0.5\linewidth]{./media/FPGA.png}
\includegraphics[width=0.5\linewidth]{./media/LUT.png}\\

\includegraphics[width=0.5\linewidth]{./media/FF-LUT.png}
\includegraphics[width=0.5\linewidth]{./media/programmed-FPGA.png}


\begin{definition}[Random Access Memory (RAM)] Structure of a RAM is as follows:\\
  \includegraphics[width=0.3\linewidth]{./media/memory-array.png}
  \includegraphics[width=0.3\linewidth]{./media/memory-array-32kb.png}
  \includegraphics[width=0.3\linewidth]{./media/SRAM-cell.png}\\
  \includegraphics[width=0.7\linewidth]{./media/SRAM.png}
\end{definition}

\begin{definition}[Read Only Memory (ROM)] Structure of a ROM is as follows:\\
  \includegraphics[width=\linewidth]{./media/rom.png}
\end{definition}

\begin{example}
Draw a Multiplexer using sum of products form.
\end{example}
\vspace{10em}


\section{Logic levels and Noise Margins}

\includegraphics[width=0.8\linewidth]{fig/logic-levels-and-noise-margins.png}

\begin{definition}[Supply Voltage ($V_{DD}$/$V_{CC}$/$V_{SS}$) ]
  \color{\notescol}
  The highest DC voltage that drives a digital circuit. As chips have progressed
  to smaller transistors, $V_{DD}$ has dropped from 5V to 1.2V or even lower to
  save power.
\end{definition}

\begin{definition}[Ground Voltage ($V_{GND}$) ]
  \color{\notescol}
  The lowest DC voltage that drives a digital circuit, typically 0V.
\end{definition}

\begin{definition}[Input high ($V_{IH}$) and Input Low ($V_{IL}$) of a gate]
  \color{\notescol}
  $V_{IH}$ is the voltage level, such that an input voltage to a gate between $V_{DD}$
  and $V_{IH}$ is considered \emph{HIGH}. Similarly, input voltage to a gate
  between $V_{IL}$ and $V_{GND}$ is considered \emph{LOW}.
\end{definition}

\begin{definition}[Output high ($V_{OH}$) and Output low ($V_{OL}$) of gate]
  \color{\notescol}
  $V_{OH}$ is the voltage level, such that an output voltage to a gate between $V_{DD}$
  and $V_{OH}$ is considered \emph{HIGH}. Similarly, output voltage to a gate
  between $V_{OL}$ and $V_{GND}$ is considered \emph{LOW}.
\end{definition}

\begin{definition}[Positive logic and Negative logic]
  \color{\notescol}
  What we have considered so far is Positive logic where \emph{HIGH} voltage is
  equated to Boolean logic \emph{TRUE} or \emph{1} and \emph{LOW} is considered
  \emph{FALSE} or \emph{0}. In negative logic these are reversed. Same physical
  circuit can represent different logical circuits in positive logic and
  negative logic.
\end{definition}

\begin{definition}[Noise margins ($NM_L$ and $NM_H$) of a channel]
  \color{\notescol}
  The maximum amount of noise that can be added (or substracted ) to a channel
  without exceeding the logic level specifications of a gate.  
  $NM_L  = V_{IL} - V_{OL}$ \\
  $NM_H  = V_{OH} - V_{IH}$
\end{definition}

\begin{example}
  \includegraphics[width=0.3\linewidth]{fig/fig1.24-noise-margins.png}\\
  If $V_{DD} = 5V$ , $V_{IL} = 1.35V$, $V_{IH} = 3.15V$, $V_{OL} = 0.33V$ and
  $V_{OH} = 3.84V$ for both the ``inverters'', then what are the low and high
  noise margins? Can the circuit tolerate 1V of noise at the channel?
\end{example}
\vspace{10em}

\section{Semiconductors and Doping}
{\tiny{Not in syllabus but good to know}}

\includegraphics[width=0.5\linewidth]{fig/semiconductors-periodic-table.png}


\includegraphics[width=0.8\linewidth]{fig/fig1.26-Si-doping.png}

\section{MOSFET: Metal Oxide Field Effect Transistors }
{\tiny{Not in syllabus but good to know}}

\includegraphics[width=0.8\linewidth]{fig/fig1.29-nMOS-pMOS-transistors.png}

\includegraphics[width=0.8\linewidth]{fig/fig1.30-nMOS-transistor-operation.png}

\section{DC Transfer characteristic}

\includegraphics[width=0.6\linewidth]{fig/nMOS-not-gate.png}

\includegraphics[width=0.8\linewidth]{fig/fig1.25-dc-transfer-characteristics.png}

\includegraphics[width=0.7\linewidth]{fig/nMOS-pMOS-switches.png}


\begin{example}
  Draw a NOT gate using nMOS transistors.
\end{example}
\vspace{10em}

\begin{example}
  Draw a NOT gate using pMOS transistors.
\end{example}
\vspace{10em}

\begin{remark}
  nMOS transistors pass 0's well (output between 0 and $V_{DD} - V_t$). pMOS
  transistors pass 1's well (output between $V_t$ and $V_{DD}$).
\end{remark}
\begin{example}
Draw CMOS NOT Gate.
\end{example}
\vspace{10em}

\begin{example}
Draw a two input CMOS NAND Gate
\end{example}
\vspace{10em}


\begin{definition}[Negative logic]
\end{definition}
\vspace{5em}

\begin{example}
  Analyze the above circuit under negative logic.
\end{example}
\vspace{10em}



\begin{example}
Draw a three input NAND using CMOS.
\end{example}
\vspace{10em}


\begin{example}
Draw a three input NOR using CMOS.
\end{example}
\vspace{10em}

\begin{example}
  Draw a two input AND gate using CMOS.
\end{example}
\vspace{10em}

\includegraphics[width=0.4\linewidth]{fig/fig1.34-inverting-logic-gate.png}

\subsection{Gates with floating output}
\includegraphics[width=0.4\linewidth]{fig/fig2.42-tristate-bus.png}

\begin{definition}[Transmission gate]
  Draw a schematic of transmission gate and truth table for transmission gate.
  What is its commonly used symbol?
\end{definition}
\vspace{10em}


\begin{definition}[Tristate buffer]
  What is tristate buffer? Draw it's symbol and truth table? Where is it used?
\end{definition}
\vspace{10em}

\begin{example}
Draw a Multiplexer using transmission gates.
\end{example}
\vspace{10em}

\begin{example}
  Draw a Multiplexer using tristate buffers.
\end{example}
\vspace{10em}


\begin{definition}[Totem-pole]
  Draw a Push-pull (or Totem-pole) output NAND gate using CMOS. Can you connect
  this gate to a shared bus?
\end{definition}
\vspace{10em}

\begin{definition}[Tristate]
  Draw a Tristate output NAND gate using CMOS with an output enable (OE) input.
  Can you connect this gate to a shared bus?
\end{definition}
\vspace{10em}

\begin{definition}[Open-collector]
  Draw a open-collector output NAND gate. Can you connect this gate to a shared bus?
\end{definition}
\vspace{10em}

\section{Verilog truth tables}

\includegraphics[width=0.3\linewidth]{fig/four-state-and-operator.png}
\includegraphics[width=0.3\linewidth]{fig/four-state-or-operator.png}

\bibliography{main}
\bibliographystyle{plain}
\end{document}
