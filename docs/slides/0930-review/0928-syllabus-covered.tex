

\section{Syllabus covered}

\newcommand{\done}{\checkmark}
\begin{itemize}
  \item[\done] Binary numbers
  \item[\done] Generate minterms, maxterms, SOP canonical form and POS
    canonical forms and convert between them\\
  \item[\done]  Understand and use the laws and theorems of Boolean Algebra
  \item[\done]  Perform algebraic simplification using Boolean algebra
  \item[\done]  Simplification using K-maps
  \item[\done]  Derive sum of product and product of sums expressions for a combinational circuit
  \item[\done]  Convert combinational logic to NAND-NAND and NOR-NOR forms
  %\item[\done]  Simplification using Quine-McCluskey method, PI tables and Petrick's method
  \item Hexadecimal, Sign-magnitude, One's-complement and
    Two's complement. Conversions between them.
  \item  Design combinational circuits for positive and negative logic
  \item  Design Hazard-free two level circuits and understand Hazards in multi-level circuits
  \item Compute noise margin of one device
  \item Describe how tri-state and open-collector outputs are different from totem-pole outputs.
  \item Different between and limitations of master-slave and edge-triggered flip-flops.
  \item Compute fan out and noise margin of one device driving the same time
  \item Know the differences and similarities between PAL, PLA, and ROMs and can use each for logic design
  \item Design combinational circuits using multiplexers and decoders
  \item Analyze a sequential circuit and derive a state-table and a state-graph
  \item Understand the difference between synchronous and asynchronous inputs
  \item Derive a state graph or state table from a word description of the problem
  \item Reduce the number of states in a state table using row reduction and implication tables
  \item Perform a state assignment using the guideline method
  \item Implement a design using JK, SR, D or T flip-flops
  \item Analyse and design both Mealy and Moore sequential circuits with multiple inputs and multiple outputs
  \item Convert between Mealy and Moore designs
  \item Partition a system into multiple state machines
\end{itemize}

\subsection{Labs (not questioned in exams)}
\begin{itemize}
  \item Use computer tools to enter designs graphically and HDL
  \item Simulate designs using computer tools
  \item Use computer tools to program gate arrays logic and debug and test
\end{itemize}

\newpage
