
\section{Objectives}
\begin{enumerate}
  \item Analyse and design both Mealy and Moore sequential circuits with multiple inputs and multiple outputs
  \item Convert between Mealy and Moore designs
  % \item Reduce the number of states in a state table using row reduction and
  %   implication tables
  %  \item Perform a state assignment using the guideline method
  %  \item Partition a system into multiple state machines
\end{enumerate}

\section{Mealy vs Moore Finite State Machines}
\begin{definition}[Finite State Machines (FSM)]~\cite[Sec~3.4]{harris2022digital}
\end{definition}
\vspace{5em}

\begin{definition}[Mealy FSM]~\cite[Sec~3.4.3]{harris2022digital}
\end{definition}
\vspace{5em}

\begin{definition}[Moore FSM]~\cite[Sec~3.4.3]{harris2022digital}
\end{definition}
\vspace{5em}


% \begin{example}
%   A sequential circuit has one input (X) and one output (Z). The circuit
%   examines groups of four consecutive inputs and produces an output Z=1 if the
%   input sequence 0010 or 0001 occurs. The sequences can overlap. Draw both Mealy
%   and Moore timing diagrams. Find the Mealy and Moore state graph.
% \end{example}
% \vspace{20em}
% 
% \begin{prob}
%   A sequential circuit has one input (X) and one output (Z). The circuit
%   examines groups of four consecutive inputs and produces an output Z=1 if the
%   input sequence 0101 or 1001 occurs. The circuit resets after every four
%   inputs. Draw both Mealy and Moore timing diagrams. Find the Mealy and Moore state graph.
% \end{prob}
% \vspace{20em}

\section{State reduction via Implication tables}
%\todo{Skip this section}

\begin{wrapfigure}[16]{r}[0pt]{0.4\linewidth}
\includegraphics[width=\linewidth]{./media/state-reduc-ex-state-table.png}
\caption{Example state table}
\end{wrapfigure}
To minimize the number of states, we will identify “equivalent
states” and eliminate any redundancy found. Two states are
equivalent if they have equivalent next states and the same output
for each possible input condition. To find equivalent states we will
create an “implication table” which looks at pairs of states and
identifies which states have to be equivalent if this pair is to be
equivalent. We can use a table to hold information about each pair.

\begin{wrapfigure}[10]{r}[0pt]{0.4\linewidth}
  \includegraphics[width=\linewidth]{./media/blank-implication-table.png}
  \caption{Implication table}
\end{wrapfigure}
To investigate all possible pairs, we could use a square table
such as this to record information about pairs of states. But
note every pair represented in the upper right “triangle” of the
table is also listed in the lower left “triangle” of the table.
Furthermore, the diagonal of the table will only present
information about a state being equivalent to itself. Only the
part of the table in bold is needed to investigate all possible
pairs of states:

\begin{wrapfigure}[5]{r}[0pt]{0.3\linewidth}
  \includegraphics[width=\linewidth]{./media/selected-implication-table.png}
  \caption{Selected implication table}
\end{wrapfigure}

\newpage
\section{State assignment}
%\todo{Skip this section}



