\section{More Gates and notations summary}
\rotatebox[origin=c]{90}{
\begin{tabular}{lccp{0.2\linewidth}cp{0.2\linewidth}}
  \toprule
  Name & C/Verilog & Boolean expr. & Truth Table & (ANSI) symbol & K-map \\
  \midrule
  NAND Gate &
  \texttt{Q = \~{}(x1 \& x2)} &
  $Q = \overline{x_1 \cdot x_2} = \overline{x_1 x_2}$ &
 \mbox{\begin{tabular}{cc|c}
 \toprule
 $x_1$ & $x_2$ & $\overline{x_1 \cdot x_2}$ \\
 \midrule
 0 & 0 & 1 \\
 0 & 1 & 1 \\
 1 & 0 & 1 \\
 1 & 1 & 0 \\
 \bottomrule
 \end{tabular}} &
\includegraphics[width=0.2\linewidth]{NAND_ANSI_Labelled.pdf} &
\begin{minipage}[b][][t]{\linewidth}
\begin{Karnaughquatre}
\minterms{0,1,2}
\maxterms{3}
\end{Karnaughquatre}
\end{minipage}
\\
  NOR Gate &
\texttt{Q = \~{}(x1 | x2)} &
$Q = \overline{x_1 + x_2}$ &
\mbox{\begin{tabular}{cc|c}
  \toprule
  $x_1$ & $x_2$ & $\overline{x_1 + x_2}$ \\
  \midrule
  0 & 0 & 1 \\
  0 & 1 & 0 \\
  1 & 0 & 0 \\
  1 & 1 & 0 \\
  \bottomrule
\end{tabular}} &
\includegraphics[width=0.2\linewidth]{NOR_ANSI_Labelled.pdf} & 
                                                               \begin{minipage}[b][][t]{\linewidth}
\begin{Karnaughquatre}
\minterms{3}
\maxterms{0,1,2}
\end{Karnaughquatre}
\end{minipage}
\\
XOR Gate &
\texttt{Q = x1 \^{} x2}
&
$ Q = x_1 \oplus x_2$
&
\mbox{\begin{tabular}{cc|c}
\toprule
$x_1$ & $x_2$ & $x_1 \oplus x_2$ \\
\midrule
0 & 0 & 0 \\
0 & 1 & 1 \\
1 & 0 & 1 \\
1 & 1 & 0 \\
\bottomrule
        \end{tabular}}
&
\includegraphics[width=0.2\linewidth]{XOR_ANSI_Labelled.pdf} &
\begin{minipage}[b][][t]{\linewidth}
\begin{Karnaughquatre}
\minterms{1,2}
\maxterms{0,3}
\end{Karnaughquatre}
\end{minipage}
\\
XNOR Gate &
\texttt{Q = \~{}(x1 \^{} x2)}
&
$ Q = \overline{x_1 \oplus x_2}$
&
\mbox{\begin{tabular}{cc|c}
        \toprule
        $x_1$ & $x_2$ & $\overline{x_1 \oplus x_2}$ \\
        \midrule
        0 & 0 & 1 \\
        0 & 1 & 0 \\
        1 & 0 & 0 \\
        1 & 1 & 1 \\
        \bottomrule
      \end{tabular}}
    &
    \includegraphics[width=0.2\linewidth]{XNOR_ANSI_Labelled.pdf} &
    \begin{minipage}[b][][t]{\linewidth}
      \begin{Karnaughquatre}
        \minterms{0,3}
        \maxterms{1,2}
      \end{Karnaughquatre}
    \end{minipage}
  \end{tabular}
}
\newpage

\section{Boolean Algebra}

\subsection{Axioms of Boolean algebra}

\begin{enumerate}
\item 
  $ 0 \cdot 0 = 0 $
\item 
    $ 1 + 1 = 1 $
\item
    $ 1 \cdot 1 = 1 $
\item
    $ 0 + 0 = 0 $
\item
    $ 0 \cdot 1 = 1 \cdot 0 = 0 $
\item
      $ \bar{0} = 1 $
\item
        $ \bar{1} = 0 $
\item
      $ x = 0 \text{ if } x \ne 1$ 
    \item
      $ x = 1 \text{ if } x \ne 0$ 
\end{enumerate}

\subsection{Single variable theorems}

\begin{enumerate}
\item $ x \cdot 0 = 0 $
  \vspace{5em}
\item $ x + 1 = 1 $
  \vspace{5em}
\item $ x \cdot 1 = x $
  \vspace{5em}
\item $ x + 0 = x $
  \vspace{5em}
\item $ x \cdot x = x $
  \vspace{5em}
\item $ x + x = x $
  \vspace{5em}
\item $ x \cdot \bar{x} = 0 $
  \vspace{5em}
\item $ x + \bar{x} = 1 $
  \vspace{5em}
\item $\bar{\bar{x}} = x $
  \vspace{5em}
\end{enumerate}

\begin{remark}[Duality]
  Swap $+$ with $\cdot$ and 0 with 1 to get another theorem
\end{remark}

\subsection{Two and three variable properties}

\begin{enumerate}
\item Commutative: $x\cdot y = y \cdot x$ , $x + y = y + x$
  \vspace{10em}
\item Associative: $x\cdot(y\cdot z) = (x \cdot y) \cdot z$, $x+(y+ z) = (x + y) + z$
  \vspace{10em}
\item Distributive: $x\cdot(y + z) = x \cdot y + x \cdot z$, $x + y \cdot z = (x + y) \cdot (y + z)$
  \vspace{10em}
\item Absorption: $x + x\cdot y = x$, $x \cdot (x+y) = x$
  \vspace{10em}
\item Combining: $x \cdot y + x \cdot \bar{y}$, $(x+y) \cdot (x + \bar{y}) = x$
  \vspace{10em}
\item DeMorgan's theorem: $\overline{x \cdot y} = \bar{x} + \bar{y}$,
  $\overline{x + y} = \bar{x} \cdot \bar{y}$.
  \vspace{10em}
\item Concensus:
  \begin{enumerate}
  \item $x + \bar{x}\cdot y = x + y$
    \vspace{10em}
  \item $x \cdot (\bar{x} + y) = x \cdot y$
    \vspace{10em}
  \item $x \cdot y + y\cdot z + \bar{x} \cdot z = x\cdot y + \bar{x} \cdot z$
    \vspace{10em}
  \item $(x + y) \cdot (y+ z) \cdot (\bar{x} + z) = (x+ y) \cdot (\bar{x} + z)$
    \vspace{10em}
  \end{enumerate}
\end{enumerate}

\begin{example}[Multiplexer]
  Multiplexer is a circuit used to select one of the input lines $x_1$ and $x_2$
  based only select input $s$. When $s=0$, $x_1$ is selected, $x_2$ is selected otherwise.
  Find a boolean expression and a circuit for multiplexer\\
  \includegraphics[width=0.2\linewidth]{multiplexer-symbol.png}
  \includegraphics[width=0.2\linewidth]{multiplexer-spec.png}
\end{example}
\vspace{10em}

\begin{example}
  Simplify $f = \bA\bB\bC + A \bB\bC + A\bB\bC $ using boolean algebra.
\end{example}
\vspace{10em}

\begin{example}
  Simplify $f = \bA\bA\bC + \bA\bB C $ using K-maps.
\end{example}
\vspace{10em}